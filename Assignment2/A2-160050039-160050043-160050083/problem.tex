\title{Assignment 1: CS 763}
\author{
  Sai Charith \\ 160050083
  \and
  Sanchit Jain\\ 160050043  
  \and
  Mayank Singhal\\ 160050039 
}

\documentclass[a4paper]{article}
\usepackage{amsmath}

\usepackage{amsfonts,amssymb,amsthm}
\usepackage[margin=0.94in]{geometry}
\usepackage{graphicx}
\begin{document}
\maketitle
%\makebox[\linewidth]{\rule{\paperwidth}{0.4pt}}
%\noindent\rule{12cm}{0.4pt}
\hrulefill

\textbf{\newline Problem 1:}

The Observed values of average MSE for \\
q1 is 0.3344 \\q2 is  0.1795 \\
q3 is 0.6752 \\

which are considerably small compared to the actual value of the x and y coordinates (between 100 and 1000)

The visualization of the points and images can be found in input folder and can be visualized using the \textbf{visualize\_image.m} script which calls impixel info.

Note the images were taken from SL3 where we found the walls with perpendicular surfaces stuck with checkerboard pattern


 The reason for normalizing is not clear from the above example as the coordinates are well behaved. If i run it without normalizing
and also commenting line 88 then also we get almost the same MSE.


 It is a good Idea to Normalize because Then we work with the \textbf{same order
 of magnitude} of the points. this not only helps in \textbf{controlling precision
 errors} but also keeps the numbers small enough to be well \textbf{in range} for
 our computation.

 So normalization is essential not only for numerical stability, 
 but also for more accurate estimation in presence of noise and faster 
 solution (in case of iterative solver).

 The precise answer as we got from \\
 https://dsp.stackexchange.com/questions/10887/why-normalize-the-data-set-before-applying-direct-linear-transform?fbclid=IwAR0gwxXtF9ire5mFa13hjVlCnn7wPyD3-FFtTSMZWdc3uT6pGFOAvpSSH5s
 \\
 is that We decrease the condition number of the Matrix M which is the
 12*12 matrix we use in the computation according to the code.
 
 
 \vspace{10pt}
 In more Detail:-\vspace{10pt}
 \\
 The source and target coordinate data are usually noisy. Without normalization, the data from source would can have two orders of magnitude larger variance than from target (or vice versa).
 
 The homography estimation usually finds parameters in a least-squares sense - hence the best statistical estimate is found only if variances of the parameters are the same (or known beforehand, but it is more practical just to normalize the input).
 
 Direct solvers do not like poorly scaled problems because numerical instabilities appear (e.g. dividing very large number by a very small number easily leads to numerical overflow).
 
 Iterative solvers struggle with badly conditioned matrices by needing more iterations.


\hrulefill \\




\textbf{\newline Problem 3:}

\begin{figure}[ht!]
 \centering
\caption{Picture with pixel values}
\includegraphics[width=1\textwidth]{Q2/input/measured.jpg}
\end{figure}

Let G' denote point in the real world corresponding to G in the image, 
 Using cross ratios we have 

\begin{equation*} 
\begin{split}
 \frac{AX*PC}{AC*PX} &= \frac{A'X'*P'C'}{A'C'*P'X'} \\
\ &= \frac{(A'P'+P'X')*(P'X'+X'C')}{(A'P'+P'X'+X'C')*(P'X')} \\
 \end{split}
\end{equation*}

Using pixel values from image and using symmetry of football field (ie $A'P' = X'C' = 18$ yards) and replacing $P'C'$ by $x$ we have

\begin{equation*} 
\begin{split}
\implies & 1.036 = \frac{(18+x)*(x+18)}{(18+x+18)*x}\\
\implies & x = 78.5\\
\therefore \quad & A'X' = 18+x+18 \\
\implies & A'X' =  114.5 \quad (length \; of \; field)\\
 \end{split}
\end{equation*}

Similarly
\begin{equation*} 
\begin{split}
 \frac{SB*AR}{SR*AB} &= \frac{S'B'*A'R'}{S'R'*A'B'} \\
\ &= \frac{(S'A'+A'B')*(A'B'+B'R')}{(S'A'+A'B'+B'R')*(A'B')} \\
 \end{split}
\end{equation*}

Using pixel values from image and using symmetry of football field (ie $S'A' = B'R'$) and replaicing $S'A',B'R'$ by $x$  we have

\begin{equation*} 
\begin{split}
\implies & 1.07 = \frac{(x+44)*(44+x)}{(44+2x)*44}\\
\implies & x = 14.7\\
\because \quad & S'R' = x+44+x \\
\implies & S'R' = 73.4 \quad (length \; of \; field)\\
 \end{split}
\end{equation*}

So, the dimensions of the pitch are $114.5 \times 73.4$

The above method uses symmtry of football field (i.e both Dees have same dimemsions. The dimensions of the field can also be calculated by calculating the vanishing points along AP and AB (using point of intersection of parallel lines) and using cross ratios. But the lines in the image are almost parellel so there would be lot of error in calculating points of intersection. Hence we preffered using symmetry in the field.

\hrulefill \\


\textbf{\newline Problem 5:}

Although there was high amount of noise in the second image the joint entropy turned out to be minimum when the images were almost aligned.

The joint entropy of an (image,constant image) is same as joint entropy of (image,image). As joint histograms take straight line shape in both cases.   


\begin{figure}[ht!]
 \centering
%\caption{Picture with pixel values}
\includegraphics[width=0.5\textwidth]{Q5/output/barbara_je.jpg}
\end{figure}
\begin{figure}[ht!]
 \centering
%\caption{Picture with pixel values}
\includegraphics[width=0.75\textwidth]{Q5/output/Barbara_aligned.jpg}
\end{figure}
\begin{figure}[ht!]
 \centering
%\caption{Picture with pixel values}
\includegraphics[width=0.5\textwidth]{Q5/output/noflash_je.jpg}
\end{figure}
\begin{figure}[ht!]
 \centering
%\caption{Picture with pixel values}
\includegraphics[width=0.75\textwidth]{Q5/output/noflash_aligned.jpg}
\end{figure}
\begin{figure}[ht!]
 \centering
%\caption{Picture with pixel values}
\includegraphics[width=0.5\textwidth]{Q5/output/unusual.jpg}
\end{figure}




\end{document}
